\documentclass[degn-notes.tex]{subfiles}

\begin{document}

\setcounter{chapter}{1}

\chapter{Observations of Galactic Nuclei and Supermassive Black Holes}

\section{Structure of galaxies and galactic nuclei}

\section{Techniques for weighing black holes}

\section{Supermassive black holes in the Local Group}

\section{Phenomenology}

\section{Evidence for intermediate-mass black holes}

\section{Evidence for binary and multiple supermassive black holes}

\section{Gravitational waves}

The amplitude, $h$, of a gravitational wave is a dimensionless quantity given by

\begin{displaymath}
  h \approx
  \frac{G}{c^4} \frac{\ddot{Q}}{D} \approx
  \frac{G M_Q}{c^2 D} \frac{v^2}{c^2},
%
  \tag{Merritt 2.55}
  \label{merritt:2.55}
\end{displaymath}

where $v$ is the internal velocity of the source, $M_Q$ is the portion of the source's mass (in units of mass, not just 0--1) participating in the quadrupolar motions, and $D$ is the distance to the source.

To understand the detector size needed to observe a gravitational wave, consider the ideal case, where the entire mass is participating in quadrupolar motions ($M_Q = M_{12}$), and the binary is located at a distance $D$ from the observer and $a$ from each other. Then we have $v^2 \approx G M_{12} / a$, and from (\ref{merritt:2.55}) we have

\begin{align*}
  h &\approx
  \frac{G M_{12}}{c^2 D} \frac{G M_{12}}{a c^2}
  \\ &\approx
  \frac{G^2}{c^4} M_{12}^2 a^{-1} D^{-1}
  \\ &\approx
  \num{2e-16}
    \qty(\frac{M_{12}}{10^{-8} M_{\Sun}})^2
    \qty(\frac{a}{\si{\milli\parsec}})^{-1}
    \qty(\frac{D}{100\si{\mega\parsec}})^{-1}
%
  \tag{Merritt 2.58}
  \label{merritt:2.58}
\end{align*}



\end{document}