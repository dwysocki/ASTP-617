\documentclass[degn-notes.tex]{subfiles}

\begin{document}

\setcounter{chapter}{1}

\chapter{Observations of Galactic Nuclei and Supermassive Black Holes}

\section{Structure of galaxies and galactic nuclei}

\subsection{Intensity profiles}

Intensity profiles are functions used to describe the intensity of a galaxy as a function of distance from the center, $I(R)$.

The \textbf{S{\'e}rsic profile} is commonly used, as it is simple and effective. The general form is:
%
\begin{displaymath}
  \ln I(R) = \ln I_e - b \, n \qty[ \qty(R/R_e)^{1/n} - 1 ].
%
  \tag{Merritt 2.3}
  \label{merritt:2.3}
\end{displaymath}
%
The \textbf{S{\'e}rsic index}, $n$, characterizes the shape of the function, and in practice, $n \in \intcc{0.5, 8}$. $b$ is typically chosen such that $R_e$, the \textbf{effective radius}, contains half of the total light. The effective intensity, $I_e$, is thus defined as $I_e \equiv I(R_e)$.

To make the function a little more intuitive, one may use differentiation to put it in the form:
%
\begin{displaymath}
  \dv{\ln I}{\ln R} = -\frac{b}{n} \qty( \frac{R}{R_e} )^{1/n}
%
  \tag{Merritt 2.4}
  \label{merritt:2.4}
\end{displaymath}
%
This form illustrates the fact that the slope on a $\log$--$\log$ plot of the intensity grows with $R$ by $(R/R_e)^{1/n}$.

In many galaxies, within a radius $R_b$, there is a break in the intensity profile (shown in Merritt Figure 2.2). For this reason, it is necessary to define a \textbf{core-S{\'e}rsic profile}:
%
\begin{displaymath}
  I(r) =
  \begin{dcases}
    I_b \qty( \frac{R_b}{R} )^\Gamma,
    & R \leq R_b,
    \\
    I_b \exp(b(R_b/R_e)^{1/n}) \exp(-b(R/R_e)^{1/n}),
    & R > R_b.
  \end{dcases}
%
  \tag{Merritt 2.8}
  \label{merritt:2.8}
\end{displaymath}


\section{Techniques for weighing black holes}

In most cases, the determination of SBH mass is done through dynamical techniques, i.e. techniques which use the motions of gas and stars to infer the black hole mass.

An important concept is the ``influence radius'', $r_m$, which is typically defined to satisfy
%
\begin{displaymath}
  M_\star(r < r_m) = 2 M_\bullet.
%
  \tag{Merritt 2.11}
  \label{merritt:2.11}
\end{displaymath}
%
In words, $r_m$ is the radius in which the enclosed mass in stars is twice that of the black hole, and the total mass is $3 M_\bullet$. The choice of 2 is somewhat arbitrary, though it is a convenient choice because of the definition of $r_h$ below. This definition, however, is only applicable to spherical galaxies, thereby leading to the alternate definitio of influence radius, $r_h$.

The alternative influence radius, $r_h$, is determined dynamically. It is defined such that the velocity of a circular orbit about the SBH from that distance, $v_c = \sqrt{G M_\bullet / r_h}$, is equal to $\sigma = v_{\rm rms} / \sqrt{3}$.
%
\begin{displaymath}
  r_h \equiv
  \frac{G M_\bullet}{\sigma^2} \approx
  10.8 \qty(\frac{M_\bullet}{10^8 \si{\solarmass}})
       \qty(\frac{\sigma}{\SI{200}{\kilo\meter\per\second}})^{-2} \si{\parsec}
%
  \tag{Merritt 2.12}
  \label{merritt:2.12}
\end{displaymath}
%
$\sigma$ is typically taken to be the rms, line-of-sight velocity of stars within the aperature, which should be centered on the SBH. This is called the \textbf{aperture dispersion}.
%
\begin{equation}
  \sigma^2 = \expval{v_{\mathrm{los}}^2}
%
  \label{eqn:aperture-dispersion}
\end{equation}


\subsection{Primary mass determination methods: Stellar and gas kinematics}

A common error is determining the mass of an SBH without being able to resolve the influence radius, $r_h$. The telescope must have an angular resolution exceeding
%
\begin{displaymath}
  \theta \lesssim
  \frac{r_h}{D} \approx
  0.02'' \qty(\frac{M_\bullet}{10^8 \si{\solarmass}})
         \qty(\frac{\sigma}{\SI{200}{\kilo\meter\per\second}})^{-2}
         \qty(\frac{D}{\SI{10}{\mega\parsec}})^{-1}
%
  \tag{Merritt 2.20}
  \label{merritt:2.20}
\end{displaymath}
%
If the influence radius is well resolved, one expects a \textbf{Keplerian rise} in velocities of stars ($v^2 = G M_\bullet / r$), leading to the definition of $r_h$ above. One should be skeptical of claims of resolving the influence radius and SBH mass in the literature. See Figure 2.5 in the text, which shows how bad some of the published data are.

In the case of galaxies with spherical symmetry, the centripetal motion of the gas near the SBH is
%
\begin{displaymath}
  v_c^2(r) = \frac{G (M_\bullet + M_\star)}{r}
%
  \tag{Merritt 2.22}
  \label{merritt:2.22}
\end{displaymath}


\setcounter{subsection}{2}
\subsection{Mass determination based on empirical correlations}





\section{Supermassive black holes in the Local Group}

\section{Phenomenology}

\setcounter{subsection}{1}
\subsection{Mass--velocity dispersion relation}

The $M_\bullet$--$\sigma$ relation is an empirical correlation between the mass of a galaxy's SBH, and the velocity dispersion of the stars in that galaxy. Typically, $\sigma$ is measured as the aperture dispersion.

The correlation is very tight, and can be expressed as
%
\begin{displaymath}
  \frac{M_\bullet}{10^8 M_\Sun} =
  (1.66 \pm 0.24)
  \qty(\frac{\sigma}{\SI{200}{\kilo\meter\per\second}})^{4.86\pm0.43}
%
  \tag{Merritt 2.33}
  \label{merritt:2.33}
\end{displaymath}



\subsection{Significance of the phenomenological relations}

The tight correlation in $M_\bullet$--$\sigma$ is quite puzzling, as one would think the velocities of stars far from the SBH would not be significantly affected by its mass.

A loose explanation works two other correlations together:
%
\begin{displaymath}
  M_{\mathrm{bulge}} \propto L^{5/4}\ \&\ L \propto \sigma^4 \implies
  M_{\mathrm{bulge}} \propto \sigma^5.
%
  \tag{Merritt 2.36}
  \label{merritt:2.36}
\end{displaymath}
%
This is very close to the observed relation, but one would not expect such a tight correlation from these.

An alternative explanation involves a negative feedback mechanism. By some mechanism, a black hole stops its own growth.

Let $L$ be the energy released in growing an SBH with mass $M(t)$, and let $\eta$ be the accretion efficiency ($\approx 10^{-1}$). Then we have the expression
%
\begin{displaymath}
  L = \eta \dot{M} c^2.
%
  \tag{Merritt 2.37}
  \label{merritt:2.37}
\end{displaymath}
%
Now, this is making a major assumption. It depends on almost all of the potential energy in the gas being released in the form of light. This would not happen if the gas fell straight in, but if it were a gradual process (as an accretion disc would be), then it works.

If we assume $\dot{M}$ to be roughly constant, then we can integrate $L(t)$ to obtain the energy released during formation
%
\begin{displaymath}
  \int_0^T L \dd{t} =
  \eta M(T) c^2 =
  \eta M_\bullet c^2.
\end{displaymath}

Consider that the bulge has a potential energy $G M_{\mathrm{bulge}}^2 / R_{\mathrm{bulge}}$, which, by the virial theorem is equal to twice the kinetic energy, or $M_{\mathrm{bulge}} \sigma^2$.

If we take the ratio of the energy released and the total gravitational potential energy of the bulge, we get
%
\begin{displaymath}
  \frac{E_{\mathrm{released}}}{E_{\mathrm{bulge}}} \approx
  \frac{\eta M_\bullet c^2}{G M_{\mathrm{bulge}}^2 / R_{\mathrm{bulge}}} \approx
  \eta \frac{M_\bullet}{M_{\mathrm{bulge}}} \frac{c^2}{\sigma^2} \approx
  225
  \qty(\frac{\eta}{0.1})
  \qty(\frac{M_\bullet / M_{\mathrm{bulge}}}{10^{-3}})
  \qty(\frac{\sigma}{\SI{200}{\kilo\meter\per\second}})^{-2} \gg
  1,
%
  \tag{Merritt 2.38}
  \label{merritt:2.38}
\end{displaymath}
%
which tells us that the energy released is much greater than that stored in the bulge.

There is more than enough energy released in growing an SBH to unbind the entire bulge.
\href{http://articles.adsabs.harvard.edu/cgi-bin/nph-iarticle_query?1998A%26A...331L...1S&amp;data_type=PDF_HIGH&amp;whole_paper=YES&amp;type=PRINTER&amp;filetype=.pdf}{Silk \& Rees (1998)}
made the argument that the SBH accretes at the Eddington limit:
%
\begin{displaymath}
  L_{\mathrm{edd}} =
  \frac{4 \pi G m_p M_\bullet c}{\sigma_e} \approx
  \num{3.2e12} \qty(\frac{M_\bullet}{10^8 M_\Sun}) L_\Sun
%
  \tag{Merritt 2.39}
  \label{merritt:2.39}
\end{displaymath}
where $m_p$ is the mass of a proton.

If we ask what SBH mass would be required to produce enough energy to unbind the bulge in a single crossing time, we wind up with the result
%
\begin{displaymath}
  M_\bullet \approx
  \frac{\sigma_e \sigma^5}{4 \pi G^2 m_p c} \approx
  \num{3e5} \qty(\frac{\sigma}{\SI{200}{\kilo\meter\per\second}})^5 M_\Sun.
%
  \tag{Merritt 2.42}
  \label{merritt:2.42}
\end{displaymath}
%
This is essentially the $M_\bullet$--$\sigma$ relation, with one caveat: the constant of proportionality is too small, by roughly $10^3$.

Andrew King proposed in 2003, that momentum is the important factor here, not energy. He proposed that the gas driven by SBH radiation will cool efficiently, and that the bulk flow is momentum driven. His proposed equation of motion is
%
\begin{displaymath}
  \dv{t}(R \dot{R}) + \frac{G M_\bullet}{R} =
  -2 \sigma^2 \qty( 1 - \frac{M_\bullet}{M_\sigma} ),
%
  \tag{Merritt 2.46}
  \label{merritt:2.46}
\end{displaymath}
%
where
%
\begin{displaymath}
  M_\sigma \equiv
  \frac{f_g \sigma_e}{\pi G^2 m_p} \sigma^4 \approx
  \num{2e8}
  \qty(\frac{f_g}{0.1})
  \qty(\frac{\sigma}{\SI{200}{\kilo\meter\per\second}})^4
  M_\Sun,
%
  \tag{Merritt 2.47}
  \label{merritt:2.47}
\end{displaymath}
%
and $f_g$ is the cosmic baryon fraction (the ratio of ordinary matter to all matter). The typically assumed value is $f_g \approx 0.16$, which leaves us with a slope on the order of what we expect. However, no mention is made in the book of the discrepancy in the exponent here ($4$) versus what is observed ($4.86\pm0.43$).



\section{Evidence for intermediate-mass black holes}

\begin{table}[h]
  \centering
  \begin{tabular}{l|l}
    \textbf{Class}    & \textbf{Mass Range}
    \\ \hline\hline
    stellar-mass      & $5$--$20$ $M_\Sun$
    \\
    intermediate-mass & $10^2$--$10^6$ $M_\Sun$
    \\
    super-massive     & $>10^6$ $M_\Sun$
  \end{tabular}
  \caption{Classification of black holes by mass}
\end{table}


\section{Evidence for binary and multiple supermassive black holes}


\section*{SBH Masses in AGN}

AGN show \emph{continuum emission} from their nuclei, i.e. emission with a spectrum characteristic of hot gas at a range of temperatures. This is believed to arise from an \emph{accretion disc}.

Broad emission lines (H$\alpha$, H$\beta$, \ldots) originating from broad emission line regions (BLR) are believed to consist of gas clouds orbiting well within the influence radius, but outside the accretion disc.

We could measure the mass of the SBH within some AGN using the relation
%
\begin{equation}
  G M_\bullet = f \times R_{\mathrm{BLR}} \times v_{\mathrm{RMS}}^2,
%
  \label{eqn:agn-mass}
\end{equation}
%
where $v_{\mathrm{RMS}}$ could be measured by the Doppler effect. One issue is that $R_{\mathrm{BLR}}$ is smaller than the influence radius, and cannot be resolved. A workaround is found by
%
\begin{equation}
  R_{\mathrm{BLR}} = c \tau,
%
  \label{eqn:blr-radius}
\end{equation}
%
where $\tau$ is the time lag between a variation in the continuum near the SBH, and a variation in the BLR luminosity. This works because of the finite speed of light.



\section{Gravitational waves}

The amplitude, $h$, of a gravitational wave is a dimensionless quantity given by
%
\begin{displaymath}
  h \approx
  \frac{G}{c^4} \frac{\ddot{Q}}{D} \approx
  \frac{G M_Q}{c^2 D} \frac{v^2}{c^2},
%
  \tag{Merritt 2.55}
  \label{merritt:2.55}
\end{displaymath}
%
where $v$ is the internal velocity of the source, $M_Q$ is the portion of the source's mass (in units of mass, not just 0--1) participating in the quadrupolar motions, and $D$ is the distance to the source.

To understand the detector size needed to observe a gravitational wave, consider the ideal case, where the entire mass is participating in quadrupolar motions ($M_Q = M_{12}$), and the binary is located at a distance $D$ from the observer and $a$ from each other. Then we have $v^2 \approx G M_{12} / a$, and from (\ref{merritt:2.55}) we have
%
\begin{align*}
  h &\approx
  \frac{G M_{12}}{c^2 D} \frac{G M_{12}}{a c^2}
  \\ &\approx
  \frac{G^2}{c^4} M_{12}^2 a^{-1} D^{-1}
  \\ &\approx
  \num{2e-16}
  \qty(\frac{M_{12}}{10^{-8} M_{\Sun}})^2
  \qty(\frac{a}{\si{\milli\parsec}})^{-1}
  \qty(\frac{D}{100\si{\mega\parsec}})^{-1}
%
  \tag{Merritt 2.58}
  \label{merritt:2.58}
\end{align*}

If you measure the rate-of-change of the frequency of gravitational waves from a binary black hole, you will not get its individual mass, but instead a combination of the two, called the ``chirp mass'', which is given by
%
\begin{equation}
  M_{\mathrm{ch}} =
  \sqrt[5]{\frac{(M_1 M_2)^3}{M_1 + M_2}}.
%
  \label{eqn:chirp-mass}
\end{equation}



\end{document}